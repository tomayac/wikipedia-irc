\documentclass{sig-alternate}

\usepackage[utf8]{inputenc}
\usepackage[T1]{fontenc}
\usepackage[T1,T2A]{fontenc}
\usepackage{lmodern}
\usepackage[activate=compatibility]{microtype}

% autoref command
\usepackage[hyphens]{url}
\usepackage[pdftex,urlcolor=black,colorlinks=true,linkcolor=black,citecolor=black]{hyperref}
\def\sectionautorefname{Section}
\def\subsectionautorefname{Subsection}

\usepackage{enumitem}

% todo macro
\usepackage{color}
\newcommand{\todo}[1]{\noindent\textcolor{red}{{\bf \{TODO}: #1{\bf \}}}}

% listings and Verbatim environment
\usepackage{fancyvrb}
\usepackage{relsize}
\usepackage{listings}
\usepackage{verbatim}
\newcommand{\defaultlistingsize}{\fontsize{8pt}{9.5pt}}
\newcommand{\inlinelistingsize}{\fontsize{8pt}{11pt}}
\newcommand{\smalllistingsize}{\fontsize{7.5pt}{9.5pt}}
\newcommand{\listingsize}{\defaultlistingsize}
\RecustomVerbatimCommand{\Verb}{Verb}{fontsize=\inlinelistingsize}
\RecustomVerbatimEnvironment{Verbatim}{Verbatim}{fontsize=\defaultlistingsize}
\lstset{frame=lines,captionpos=b,numberbychapter=false,escapechar=§,
        aboveskip=2em,belowskip=1em,abovecaptionskip=0.5em,belowcaptionskip=0.5em,
        framexbottommargin=-1em,basicstyle=\ttfamily\listingsize\selectfont}

% use Courier from this point onward
\let\oldttdefault\ttdefault
\renewcommand{\ttdefault}{pcr}
\let\oldurl\url
\renewcommand{\url}[1]{\inlinelistingsize\oldurl{#1}}

\lstdefinelanguage{JavaScript}{
  keywords={push, typeof, new, true, false, catch, function, return, null, catch, switch, var, if, in, while, do, else, case, break},
  keywordstyle=\bfseries,
  ndkeywords={class, export, boolean, throw, implements, import, this},
  ndkeywordstyle=\color{darkgray}\bfseries,
  identifierstyle=\color{black},
  sensitive=false,
  comment=[l]{//},
  morecomment=[s]{/*}{*/},
  commentstyle=\color{darkgray},
  stringstyle=\color{red},
  morestring=[b]',
  morestring=[b]"
}

% linewrap symbol
\definecolor{grey}{RGB}{130,130,130}
\newcommand{\linewrap}{\raisebox{-.6ex}{\textcolor{grey}{$\hookleftarrow$}}}

\hyphenation{Wikistream Wikipedia Wikipedias}

%\def\baselinestretch{0.99}

\begin{document}

% --- Author Metadata here ---
\conferenceinfo{World Wide Web Conference}{'13 Rio de Janeiro, Brazil}
\CopyrightYear{2013} % Allows default copyright year (20XX) to be over-ridden - IF NEED BE.
%\crdata{0-12345-67-8/90/01}  % Allows default copyright data (0-89791-88-6/97/05) to be over-ridden - IF NEED BE.
% --- End of Author Metadata ---


\title{MJ no more: Using Concurrent Wikipedia Edit Spikes\\ with Social Network Plausibility Checks\\ to Detect Breaking News}

\numberofauthors{1}\author{
\alignauthor
Thomas Steiner\\
	\affaddr{Google Germany GmbH}\\
	\affaddr{ABC-Str. 19}\\
	\affaddr{20354 Hamburg, Germany}\\
	\email{tomac@google.com}
}
\maketitle
\begin{abstract}
\fontencoding{T1}\selectfont
We have developed an application called Wikipedia Live Monitor
that monitors article edits on different language versions of Wikipedia---%
as they happen in realtime.
Wikipedia articles in different languages are highly interlinked.
For example, the English article \emph{``en:2013\_Russian\_meteor\_event''}
on the topic of the February 15 meteoroid
that exploded over the region of Chelyabinsk Oblast, Russia,
is interlinked with \fontencoding{T2A}\selectfont
\emph{``ru:Падение\_метеорита\_на\_Урале\_в\_2013\_году''},
\fontencoding{T1}\selectfont
the Russian article on the same topic.
As we monitor multiple language versions of Wikipedia in parallel,
we can exploit this fact to detect \emph{concurrent edit spikes}
of Wikipedia articles covering the same topics,
both in only one and in different languages.
We treat such concurrent edit spikes as signals
for potential breaking news events, whose plausibility we then check 
with full-text cross-language searches on multiple social networks.
Unlike the reverse approach of monitoring social networks first,
and potentially checking plausibility on Wikipedia second,
the approach proposed in this paper has the advantage of
being less prone to false-positive alerts, while being equally sensitive
to true-positive events, however, at only a~fraction of the processing cost.
A~live demo of our application is available online at the URL \url{http://wikipedia-irc.herokuapp.com/},
the source code is available
under the terms of the Apache~2.0 license at 
\url{https://github.com/tomayac/wikipedia-irc}.

\end{abstract}

\category{H.3.3}{Information Search and Retrieval}{Clustering}

\terms{Algorithms}

\keywords{Breaking News Detection, Wikipedia, Social Networks}

\section{Introduction}

\subsection{Motivation}

Shortly after the celebrity news website TMZ
broke the premature news that the King of Pop Michael Jackson~(MJ) had died,%
\footnote{MJ dead: \url{http://www.tmz.com/2009/06/25/michael-jackson-dies-death-dead-cardiac-arrest/},
accessed 02/18/2013}
the Internet slowed down.%
\footnote{Internet slow-down: \url{http://news.bbc.co.uk/2/hi/technology/8120324.stm}, accessed 02/18/2013}
Initially, Wikipedia's website administrators started noting abnormal load spikes~%
\cite{vibber2009currentevents}. Shortly afterwards, caching issues
caused by a~so-called edit war~\cite{beaumont2009editwar} led the site to go down:
Wikipedia editors worldwide made concurrent edits
to the Michael Jackson Wikipedia article, doing and undoing changes
regarding the tense of the article, death date,
and the circumstances of the at the time officially still unconfirmed fatality.
While Wikipedia engineers have worked hard
to ensure that future load spikes
do not take the site down again, there is without dispute a~lot of research potential
in analyzing such editing activity.

\subsection{Hypotheses and Research Questions}

In this paper, we present an application that monitors article edits
of different language versions of Wikipedia in realtime
in order to detect concurrent edit spikes that may be the source of
breaking news events.
When a~concurrent edit spike has been detected,
we use full-text searches on social networks
as plausibility checks to filter out false-positive alerts.
We are driven by the following hypotheses.

\begin{itemize}
  \itemsep0em
  \item[(H1)] Breaking news events spread over social networks,
    independent from where the news broke initially.
  \item[(H2)] If a~breaking news event is important, it will be reflected on
    at least one language edition of Wikipedia.
  \item[(H3)] The time between when the news broke first and the news
    being reflected on Wikipedia is considerably short.   
\end{itemize}

\noindent These hypotheses lead us to the research questions below.

\begin{itemize}
  \itemsep0em
  \item[(Q1)] Can concurrent Wikipedia edit spikes combined with
    social network plausibility checks capture major breaking news events,
    and if so, with what delay?
  \item[(Q2)] Is the approach \emph{Wikipedia first, social networks second}
    at least as powerful as the reverse approach?
\end{itemize}

In this paper, we do not answer all research questions yet,
however, lay the foundation stone for future research in this area
by introducing the Wikipedia Live Monitor application.

\section{Related Work}

We refer to an event as breaking news, if the event is of significant importance
to a~considerable amount of the population.
Petrovi\'{c} \emph{et~al.} define~\cite{petrovic2010streamingfirststory}
the goal of new event detection (or first story detection) as
\textit{``given a~sequence of stories, to identify the first story
to discuss a~particular event.''}
They define an event as \textit{``something that happens
at some specific time and place.''}
Classic streaming analysis of social network microposts so far has been mainly
focused on Twitter, a~microblogging social network that provides access
to a~sampled stream of generated microposts by means of its Streaming API.%
\footnote{Twitter Streaming API: \url{https://dev.twitter.com/docs/api/1.1/get/statuses/sample},
accessed 02/18/2013}
Petrovi\'{c} \emph{et~al.} explain~\cite{petrovic2010streamingfirststory}:
\textit{``in the streaming model of computation,
items arrive continuously in a~chronological order, and have to be
processed in bounded space and time.''}
In the referenced paper, the authors report on a~system for streaming
new event detection applied to Twitter based on locality sensitive hashing.
Hu \emph{et~al.} provide an analysis of how news break and spread on Twitter~%
\cite{hu2012breakingnews}.
The task of linking news events with social media is covered by Tsagkias
\emph{et~al.} in~\cite{tsagkias2011linkingonlinenews}.
With this paper, we stand on the shoulders\footnote{Hence the title of this paper.}
of Osborne \emph{et~al.}~%
\cite{osborne2012bieber}, who use Wikipedia page view statistics%
\footnote{Page view statistics for Wikimedia projects: \url{http://dumps.wikimedia.org/other/pagecounts-raw/},
accessed 02/18/2013}
as a~means to filter spurious events
stemming from event detection over social network streams.
Our approach reverses theirs, however, instead of the only hourly updated
page view statistics, we use realtime change notifications,
as explained in \autoref{sec:wikipedia-recent-changes}.
Wikipedia Live Monitor is partly based on an application called
Wikistream, developed by Ed Summers \emph{et~al.}, which was described in~%
\cite{summers2011odetonode}.

\section{Implementation Details}

\subsection{Wikipedia Recent Changes}
\label{sec:wikipedia-recent-changes}

As described earlier, our application monitors concurrent edit spikes
on different language versions of Wikipedia.
In the current implementation, we monitor 42 different Wiki\-pedias, 5 with 
$\geq$~1,000,000 and 37 with $\geq$~100,000 articles.%
\footnote{List of Wikipedias by size: \url{http://meta.wikimedia.org/wiki/List_of_Wikipedias},
accessed 02/18/2013}
Changes to any single one article are communicated by a~chat bot
over Wikipedia's own Internet Relay Chat~(IRC) server (\url{irc.wikimedia.org}),%
\footnote{Raw IRC feeds of recent changes: \url{http://meta.wikimedia.org/wiki/IRC/Channels\#Raw_feeds}, accessed 02/18/2013}
so that parties interested in the data can listen to the changes as they happen.
For each language version, there is a~specific chat room following the pattern
\texttt{"\#" + language + ".wikipedia"}.
For example, changes to Russian Wikipedia articles will be streamed to the room
\texttt{\#ru.wikipedia}.
A~special case is the room \texttt{\#wikidata.wikipedia} for Wikidata~%
~\cite{vrandecic2012wikidata},
a~platform for the collaborative acquisition and maintenance
of structured data to be used by
Wikimedia projects like Wikipedia.
A~sample chat message with the components separated
by the asterisk character \texttt{'*'}
announcing a~change can be seen in the following.
\texttt{"[[Juniata River]] http://en.wikipedia.org/w/index.php?diff=516269072&oldid=5146\-59029 * Johanna-Hypatia * (+67) Category:Place names of Native American origin in Pennsylvania"}.
The message components are (i) article name, (ii) revision URL,
(iii) Wikipedia editor handle, and (iv) change size and change description.

\subsection{Article Clusters}

We cluster edits of articles about the same topic
but written in different languages in article clusters.
The example of the English
\emph{``en:2013\_Russian\_meteor\_event''}
and the Russian article \fontencoding{T2A}\selectfont
\emph{``ru:Падение\_метеорита\_на\_Урале\_в\_2013\_году''}
\fontencoding{T1}\selectfont that are both in the same cluster illustrate this.
We use the Wikipedia API to retrieve language links for a~given article.
The URL pattern for the API is as follows.
\url{http://$LANGUAGE.wikipedia.org/w/api.php?action=query&format=json&-prop=langlinks&titles=$ARTICLE}.


\subsection{Comparing Article Revisions}

The Wikipedia API provides means to retrieve the actual changes
that were made during an edit including additions, deletions,
and modifications, in a~\texttt{diff}-like manner.
The URL pattern is as follows.
\url{http://$LANGUAGE.wikipedia.org/w/api.php?action=compare&torev=$TO&fromrev=$FROM&format=json}. %$%
This allows us to classify edits in categories, like, \emph{e.g.},
negligible trivial edits (punctuation correction) and 
major important edits (new paragraph for an article),
which allows us to disregard seemingly concurrent edits
in order to avoid false-positive alerts.

\subsection{Breaking News Criteria}

Our application Wikipedia Live Monitor puts  
detected article clusters in a~monitoring loop in which they remain
until their time-to-live (240~seconds) is over.
In order for an article cluster in the monitoring loop
to be identified as breaking news candidate,
the following breaking news criteria have to be fulfilled.

\begin{description}
  \itemsep0em
  \item[$\geq$~5~Occurrences:] An article cluster must have occurred
  in at least 5~edits.
  \item[$\leq$~60~Seconds Between Edits:] An article cluster may have
  at maximum 60~seconds in between edits.
  \item[$\geq$~2~Concurrent Editors:] An article cluster must have been edited
  by at least 2~concurrent editors.
  \item[$\leq$~240~Seconds Since Last Edit:] An article cluster's last edit
  may not be longer ago than 240~seconds.
\end{description}

The exact parameters of the breaking news criteria above
were determined empirically by analyzing Wikipedia edits
over several hours and repeatedly adjusting the settings until
major news events happening at the same time were detected.
The created dataset split in three chunks has been made publicly available.%
\footnote{Wikipedia Live Monitor dataset: \url{https://www.dropbox.com/sh/2qsg1zhb8p35fxf/Dghn55y0kh},
accessed 02/18/2013}

\subsection{Application Pseudocode}

The Wikipedia Live Monitor application has been implemented in Node.js,
a~server side JavaScript software system
designed for writing scalable Internet applications.
Programs are created using event-driven, asynchronous input/output operations
to minimize overhead and maximize scalability.
\autoref{code:pseudocode} shows the pseudocode of the two main event loops of the
Wikipedia Live Monitor application.
The actual implementation is based on 
Martyn Smith's Node.js IRC library%
\footnote{Node.js IRC library: \url{https://github.com/martynsmith/node-irc},
accessed 02/18/2013} and
the WebSockets API and protocol~\cite{hickson2012websockets},
wrapped by  Guillermo Rauch's library Socket.IO%
\footnote{Socket.IO library: \url{http://socket.io/},
accessed 02/18/2013}.

\begin{lstlisting}[caption=Pseudocode of the two main event loops of the Wikipedia Live Monitor application,
  label=code:pseudocode, float]
§\textbf{Input: irc, listening on Wikipedia recent changes}§ 
§\textbf{Output: breakingNewsCandidates, breaking news candidates}§ 

§\textbf{init:}§

monitoringLoop = {}
articleClusters = {}
breakingNewsCandidates = {}
  
§\textit{\# Event loop 1:}§
§\textit{\# When a new message arrives}§
irc.on.message §\textbf{do (article)}§
  langRefs = getLanguageReferences(article)
  articleRevs = getArticleRevisions(article)
  cluster = clusterArticles(article, langRefs)  
  
  §\textit{\# Create new cluster for previously unseen article}§
  §\textbf{if}§ cluster not in monitoringLoop
    monitoringLoop.push(cluster)
    articleClusters.push(cluster)
    updateStatistics(cluster)
    emit.newCluster(cluster, articleRevs)    
  §\textit{\# Update existing cluster, article was seen before}§
  §\textbf{else}§
    updateStatistics(cluster)
    emit.existingCluster(cluster, articleRevs)
    §\textit{\# Check breaking news criteria}§
    §\textbf{if}§ cluster.occurrences >= 5
      §\textbf{if}§ cluster.secsBetweenEdits <= 60      
        §\textbf{if}§ cluster.numEditors >= 2
          §\textbf{if}§ cluster.secsSinceLastEdit <= 240
            socialNetworks.search(langRefs)
            breakingNewsCandidates.push(cluster)
            emit.breakingNewsCandidate(cluster)            
          §\textbf{end if}§          
        §\textbf{end if}§  
      §\textbf{end if}§        
    §\textbf{end if}§          
  §\textbf{end if}§
  §\textbf{return}§ breakingNewsCandidates
§\textbf{end do}§

§\textit{\# Event loop 2:}§
§\textit{\# Remove too old clusters regularly}§
timeout.every.240seconds §\textbf{do}§
  §\textbf{for each}§ cluster §\textbf{in}§ monitoringLoop
    §\textbf{if}§ cluster.secsSinceLastEdit >= 240
      monitoringLoop.remove(cluster)
      articleClusters.remove(cluster)
    §\textbf{end if}§
  §\textbf{end for}§ 
§\textbf{end do}§
\end{lstlisting}

\section{Premature Evaluation}

\begin{figure*}
  \centering
  \includegraphics[width=1\linewidth]{./wikipedia-live-monitor.png}
  \caption{Screenshot with
    an article cluster of four concurrently edited articles (ru, en, pt, ca).
    All breaking news criteria are fulfilled, the cluster is a breaking news candidate.
    Cross-language social network search results for en and pt can be seen.}
  \label{fig:screenshot}
\end{figure*}


\section{Conclusions and Future Work}

\balancecolumns
\bibliographystyle{abbrv}
\bibliography{www2013devtrack}

\balancecolumns
\end{document}