\documentclass{sig-alternate}

\usepackage[utf8]{inputenc}
\usepackage[T1]{fontenc}
\usepackage[T1,T2A]{fontenc}
\usepackage{lmodern}
\usepackage[activate=compatibility]{microtype}

% autoref command
\usepackage[hyphens]{url}
\usepackage[pdftex,urlcolor=black,colorlinks=true,linkcolor=black,citecolor=black]{hyperref}
\def\sectionautorefname{Section}
\def\subsectionautorefname{Subsection}

\usepackage{enumitem}

% todo macro
\usepackage{color}
\newcommand{\todo}[1]{\noindent\textcolor{red}{{\bf \{TODO}: #1{\bf \}}}}

% listings and Verbatim environment
\usepackage{fancyvrb}
\usepackage{relsize}
\usepackage{listings}
\usepackage{verbatim}
\newcommand{\defaultlistingsize}{\fontsize{7.5pt}{9.5pt}}
\newcommand{\inlinelistingsize}{\fontsize{8pt}{11pt}}
\newcommand{\smalllistingsize}{\fontsize{7.5pt}{9.5pt}}
\newcommand{\listingsize}{\defaultlistingsize}
\RecustomVerbatimCommand{\Verb}{Verb}{fontsize=\inlinelistingsize}
\RecustomVerbatimEnvironment{Verbatim}{Verbatim}{fontsize=\defaultlistingsize}
\lstset{frame=lines,captionpos=b,numberbychapter=false,escapechar=§,
        aboveskip=0.5em,belowskip=0em,abovecaptionskip=0em,belowcaptionskip=0em,
framexbottommargin=-1em,
        basicstyle=\ttfamily\listingsize\selectfont}

% use Courier from this point onward
\let\oldttdefault\ttdefault
\renewcommand{\ttdefault}{pcr}
\let\oldurl\url
\renewcommand{\url}[1]{\inlinelistingsize\oldurl{#1}}

\lstdefinelanguage{JavaScript}{
  keywords={push, typeof, new, true, false, catch, function, return, null, catch, switch, var, if, in, while, do, else, case, break},
  keywordstyle=\bfseries,
  ndkeywords={class, export, boolean, throw, implements, import, this},
  ndkeywordstyle=\color{darkgray}\bfseries,
  identifierstyle=\color{black},
  sensitive=false,
  comment=[l]{//},
  morecomment=[s]{/*}{*/},
  commentstyle=\color{darkgray},
  stringstyle=\color{red},
  morestring=[b]',
  morestring=[b]"
}

% linewrap symbol
\definecolor{grey}{RGB}{130,130,130}
\newcommand{\linewrap}{\raisebox{-.6ex}{\textcolor{grey}{$\hookleftarrow$}}}

% more pleasing quote environment
\usepackage{tikz}
\newcommand*{\openquote}{\tikz[remember picture,overlay,xshift=-7pt,yshift=1pt]
     \node (OQ) {\fontfamily{fxl}\fontsize{16}{16}\selectfont``};\kern0pt}
\newcommand*{\closequote}{\tikz[remember picture,overlay,xshift=2pt,yshift=-4.5pt]
     \node (CQ) {\fontfamily{fxl}\fontsize{16}{16}\selectfont''};}
\renewenvironment{quote}%
{\setlength{\parindent}{1cm}\par\openquote}
{\closequote\vspace{-4.5pt}
}

% bullet numbers
\usepackage{tkz-graph}
\usetikzlibrary{matrix,arrows,decorations.pathmorphing,shapes}
\newcommand{\dobulletnumber}[1]{\node[circle,text=white,fill=gray,anchor=west,inner sep=1pt] {\sffamily #1}}
\newcommand{\bulletnumber}[1]{\tikz[baseline=-2.5,overlay]\dobulletnumber{#1};}
\newcommand{\bulletref}[1]{\tikz[baseline=-2.5]\dobulletnumber{\fontsize{8}{8}\selectfont#1};}

\hyphenation{DBpedia RESTdesc}

%\def\baselinestretch{0.99}

\begin{document}

\title{MJ no more: using concurrent Wikipedia edit spikes with social network plausibility checks to detect breaking news}

\numberofauthors{1}\author{
\alignauthor
Thomas Steiner\\
	\affaddr{Google Germany GmbH}\\
	\affaddr{ABC-Str. 19}\\
	\affaddr{20354 Hamburg, Germany}\\
	\email{tomac@google.com}
}
\maketitle
\begin{abstract}
\fontencoding{T1}\selectfont
We have developed an application called Wikipedia Live Monitor
that monitors edits on different language versions of Wikipedia---%
as they happen in realtime.
Wikipedia articles in different languages are highly interlinked.
For example, the English article \emph{``en:2013\_Russian\_meteor\_event''}
on the topic of the February 15 meteoroid
that exploded over the region of Chelyabinsk Oblast, Russia,
is interlinked with \fontencoding{T2A}\selectfont
\emph{``ru:Падение\_метеорита\_на\_Урале\_в\_2013\_году''},
\fontencoding{T1}\selectfont \emph{i.e.},
the Russian article on the same topic.
As we monitor multiple language versions of Wikipedia in parallel,
we can exploit this fact to detect \emph{concurrent edit spikes}
of Wikipedia articles covering the same topics,
both in only one and in different languages.
We treat such concurrent edit spikes as signals
for potential breaking news events, whose plausibility we then check 
with full-text cross-language searches on multiple social networks.
Unlike the opposite approach of monitoring social networks first
and potentially checking plausibility on Wikipedia second,
the approach proposed in this paper has the advantage of
being less prone to false-positive alerts while being equally sensitive
to true-positive events, however, at a~fraction of the processing cost.

\end{abstract}

\category{I.2.10}{Vision and Scene Understanding}{Video analysis} 
\category{H.5.1}{Multimedia Information Systems}{Video (e.g., tape, disk, DVI)}

\terms{Algorithms}

\keywords{}

\section{Introduction}



\cite{petrovic2010streamingfirststory}

\cite{osborne2012bieber}

\footnote{Raw IRC feeds of recent changes \url{http://meta.wikimedia.org/wiki/IRC/Channels\#Raw_feeds}, accessed 02/18/2013}
\footnote{Page view statistics for Wikimedia projects \url{http://dumps.wikimedia.org/other/pagecounts-raw/}, accessed 02/18/2013}


1) >= 5 Occurrences 
An article cluster must have at least n edits before it is considered a breaking news candidate.

2) <=60 Seconds Between Edits 
An article cluster may have at max n seconds in between edits in order to be regarded a breaking news candidate.

3) >=2 Concurrent Editors
An article cluster must be edited by at least n concurrent editors before it is considered a breaking news candidate.

4) <=240 Seconds Since Last Edit 
An article cluster is thrown out of the monitoring loop if its last edit is longer ago than n seconds.

\bibliographystyle{abbrv}
\bibliography{www2013devtrack}

\balancecolumns
\end{document}